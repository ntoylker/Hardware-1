\documentclass[12pt, a4paper]{article}

% --- ΒΑΣΙΚΑ ΠΑΚΕΤΑ ---
\usepackage[utf8]{inputenc}
\usepackage[T1]{fontenc} % <-- Πρόσθεσε αυτή τη γραμμή
\usepackage[greek]{babel}
\usepackage{geometry} % Για τα περιθώρια
\geometry{a4paper, margin=1in} % Περιθώρια 1 ίντσα
\usepackage{graphicx} % Για εισαγωγή εικόνων
\usepackage{hyperref} % Για υπερσυνδέσμους (π.χ. στον πίνακα περιεχομένων)
\hypersetup{
    colorlinks=true,
    linkcolor=blue,
    filecolor=magenta,      
    urlcolor=cyan,
    pdftitle={Αναφορά Project HW-1},
    pdfpagemode=FullScreen,
}

% --- ΠΑΚΕΤΑ ΓΙΑ ΚΩΔΙΚΑ ---
\usepackage{listings} % Για εισαγωγή κώδικα

\usepackage{xcolor} % Για χρώματα στον κώδικα

% --- Ρυθμίσεις για το 'listings' (Verilog) ---
\definecolor{codegreen}{rgb}{0,0.6,0}
\definecolor{codegray}{rgb}{0.5,0.5,0.5}
\definecolor{codepurple}{rgb}{0.58,0,0.82}
\definecolor{backcolour}{rgb}{0.96,0.96,0.96}

\lstdefinestyle{verilogstyle}{
    language=Verilog,
    backgroundcolor=\color{backcolour},   
    commentstyle=\color{codegreen},
    keywordstyle=\color{blue},
    numberstyle=\tiny\color{codegray},
    stringstyle=\color{codepurple},
    basicstyle=\ttfamily\footnotesize, % Μέγεθος γραμματοσειράς κώδικα
    breakatwhitespace=false,         
    breaklines=true,                 
    captionpos=b,                    
    keepspaces=true,                 
    numbers=left,                    % Αρίθμηση γραμμών
    numbersep=5pt,                   
    showspaces=false,                
    showstringspaces=false,
    showtabs=false,                  
    tabsize=2
}
\lstset{
    basicstyle=\ttfamily\selectlanguage{english}\footnotesize,
}


% --- ΣΤΟΙΧΕΙΑ ΕΡΓΑΣΙΑΣ ---
\title{Ψηφιακά Συστήματα \textlatin{HW} σε Χαμηλά Επίπεδα Λογικής I \\ \textlatin{Project's Report}}
\author{Νίκος Τουλκερίδης\\ ΑΕΜ: 10718}
\date{\today}




%%%%%%%%%%%%%%%%%%%%%%%%%%%%%%%%%%%%%%%%%%%%%%%%%%%%%%%%%%
%                   ΚΥΡΙΩΣ ΑΝΑΦΟΡΑ                      %
%%%%%%%%%%%%%%%%%%%%%%%%%%%%%%%%%%%%%%%%%%%%%%%%%%%%%%%%%%
\begin{document}

\maketitle % Δημιουργία σελίδας τίτλου

\newpage
\tableofcontents % Πίνακας περιεχομένων

%% -- 1 -- %%
\newpage
\section*{Εισαγωγή}\addcontentsline{toc}{section}{Εισαγωγή}

Η παρούσα εργασία υλοποιήθηκε στα πλαίσια του μαθήματος <<Ψηφιακά Συστήματα \textlatin{HW} σε Χαμηλά Επίπεδα Λογικής I>>.
Κεντρικός στόχος είναι η σχεδίαση, η υλοποίηση σε γλώσσα περιγραφής υλικού \textlatin{Verilog} και η προσομοίωση της λειτουργίας τεσσάρων διακριτών ψηφιακών κυκλωμάτων, τα οποία συνδυάζονται για να δημιουργήσουν ένα ολοκληρωμένο, αν και απλό, σύστημα.
Η εργασία αποτελείται από τέσσερα βασικά μέρη:

\begin{itemize}
    \item \textbf{Άσκηση 1:} Σχεδίαση μιας 32-\textlatin{bit} Αριθμητικής/Λογικής Μονάδας (\textlatin{ALU}), ικανής να εκτελεί 12 διαφορετικές αριθμητικές, λογικές πράξεις και πράξεις ολίσθησης.
    \item \textbf{Άσκηση 2:} Υλοποίηση μιας απλής αριθμομηχανής 16-\textlatin{bit}, η οποία χρησιμοποιεί την \textlatin{ALU} της Άσκησης 1 και έναν συσσωρευτή (\textlatin{accumulator}) για να εκτελεί διαδοχικούς υπολογισμούς.
    \item \textbf{Άσκηση 3:} Σχεδίαση ενός αρχείου καταχωρητών (\textlatin{register file}) μεγέθους 16*32-\textlatin{bit}, το οποίο διαθέτει πολλαπλές θύρες ανάγνωσης (4) και εγγραφής (2).
    \item \textbf{Άσκηση 4:} Σχεδίαση και υλοποίηση ενός μικρού επιταχυντή \textlatin{AI} (\textlatin{AI accelerator}) που μοντελοποιεί ένα απλό νευρωνικό δίκτυο.
Αυτό το τελικό σύστημα χρησιμοποιεί την \textlatin{ALU} (μέσω μιας μονάδας \textlatin{MAC}) και το \textlatin{register file} για την εκτέλεση των απαιτούμενη υπολογισμών.
\end{itemize}

Στις επόμενες ενότητες αυτής της αναφοράς παρουσιάζεται αναλυτικά η σχεδιαστική προσέγγιση που ακολουθήθηκε, ο πλήρης κώδικας \textlatin{Verilog} για κάθε \textlatin{module}, καθώς και τα αποτελέσματα της προσομοίωσης που χρησιμοποιήθηκαν για την επαλήθευση της ορθής λειτουργίας τους.

%% -- 2 -- %%
\newpage
\section{Άσκηση 1: Αριθμητική/Λογική Μονάδα (\textlatin{ALU})}

% 2.1
\subsection{Σκοπός και Προδιαγραφές}
Ο σκοπός της πρώτης άσκησης ήταν η σχεδίαση και υλοποίηση σε \textlatin{Verilog} μιας Αριθμητικής/Λογικής Μονάδας (\textlatin{ALU}) 32-\textlatin{bit}.
Η μονάδα αυτή θα αποτελέσει δομικό στοιχείο για τις επόμενες ασκήσεις, συγκεκριμένα την αριθμομηχανή και τον επιταχυντή \textlatin{AI}.
Βάσει των προδιαγραφών, η \textlatin{ALU} έπρεπε να σχεδιαστεί ως ένα αμιγώς \textbf{συνδυαστικό κύκλωμα}(\textlatin{combinational circuit}).
Αυτό σημαίνει ότι οι έξοδοί της εξαρτώνται αποκλειστικά από τις τρέχουσες τιμές των εισόδων και δεν υπάρχει μνήμη κατάστασης, ούτε ανάγκη για σήμα ρολογιού ή επαναφοράς.
Οι θύρες εισόδου και εξόδου της μονάδας, όπως καθορίστηκαν στον πίνακα της εκφώνησης, είναι:
\begin{itemize}
    \item \textbf{Είσοδοι:} \texttt{\textlatin{op1}} (32-\textlatin{bit}) και \texttt{\textlatin{op2}} (32-\textlatin{bit}) ως οι δύο προσημασμένοι τελεστές σε μορφή συμπληρώματος ως προς 2, και \texttt{\textlatin{alu\_op}} (4-\textlatin{bit}) ως σήμα ελέγχου για την επιλογή της εκτελούμενης πράξης.
    \item \textbf{Έξοδοι:} \texttt{\textlatin{result}} (32-\textlatin{bit}) που φέρει το αποτέλεσμα της πράξης, \texttt{\textlatin{zero}} (1-\textlatin{bit}) που ενεργοποιείται (γίνεται 1) όταν το \texttt{\textlatin{result}} είναι μηδέν, και \texttt{\textlatin{ovf}} (1-\textlatin{bit}) που σηματοδοτεί υπερχείλιση (\textlatin{overflow}) για τις αριθμητικές πράξεις (πρόσθεση, αφαίρεση, πολλαπλασιασμός).
\end{itemize}

Η \textlatin{ALU} σχεδιάστηκε για να υποστηρίζει 12 διαφορετικές πράξεις, οι οποίες επιλέγονται από την είσοδο \texttt{\textlatin{alu\_op}} και ομαδοποιούνται ως εξής:
\begin{itemize}
    \item \textbf{Αριθμητικές (Προσημασμένες):} Πρόσθεση (\textlatin{4'b0100}), Αφαίρεση (\textlatin{4'b0101}), Πολλαπλασιασμός (\textlatin{4'b0110}).
    \item \textbf{Λογικές:} \textlatin{AND} (\textlatin{4'b1000}), \textlatin{OR} (\texttt{4'b1001}), \textlatin{NOR} (\textlatin{4'b1010}), \textlatin{NAND} (\textlatin{4'b1011}), \textlatin{XOR} (\textlatin{4'b1100}).
    \item \textbf{Ολισθήσεις:} Λογική Δεξιά/Αριστερά (\textlatin{4'b0000}, \textlatin{4'b0001}) και Αριθμητική Δεξιά/Αριστερά (\textlatin{4'b0010}, \textlatin{4'b0011}).
\end{itemize}
Όπως απαιτήθηκε, οι 4-\textlatin{bit} κωδικοί για αυτές τις πράξεις ορίστηκαν ως σταθερές \texttt{\textlatin{parameter}} εντός του \textlatin{module} \texttt{\textlatin{alu.v}} για βελτιωμένη αναγνωσιμότητα.

% 2.2
\subsection{Υλοποίηση (\textlatin{alu.v})}
Η υλοποίηση της \textlatin{ALU} (αρχείο \texttt{\textlatin{alu.v}}) βασίστηκε σε ένα κεντρικό συνδυαστικό μπλοκ \texttt{\textlatin{always @(*)}}.
Εντός αυτού του μπλοκ, ένας πολυπλέκτης υλοποιήθηκε χρησιμοποιώντας μια δομή \texttt{\textlatin{case}}.
Αυτή η δομή ελέγχει την είσοδο \texttt{\textlatin{alu\_op}} και επιλέγει την κατάλληλη λογική για τον υπολογισμό του \texttt{\textlatin{result}} και του \texttt{\textlatin{ovf}}.
Οι είσοδοι \texttt{\textlatin{op1}} και \texttt{\textlatin{op2}} δηλώθηκαν ως \texttt{\textlatin{signed}}, κάτι που είναι κρίσιμο για να εξασφαλιστεί ότι η \textlatin{Verilog} θα εκτελέσει τις αριθμητικές πράξεις και την αριθμητική ολίσθηση δεξιά ως προσημασμένες.
Ιδιαίτερη προσοχή δόθηκε στον ορθό υπολογισμό της υπερχείλισης (\texttt{\textlatin{ovf}}) για τις τρεις προσημασμένες αριθμητικές πράξεις, χρησιμοποιώντας ενδιάμεσα \textlatin{wires} για σαφήνεια:
\begin{itemize}
    \item \textbf{Πρόσθεση (\textlatin{SUM}):} Η υπερχείλιση ανιχνεύεται όταν οι δύο τελεστές έχουν το ίδιο πρόσημο (\textlatin{bit} 31) και το αποτέλεσμα (\texttt{\textlatin{add\_res}}) έχει διαφορετικό πρόσημο από αυτούς.
    \item \textbf{Αφαίρεση (\textlatin{SUB}):} Η υπερχείλιση ανιχνεύεται όταν οι τελεστές έχουν διαφορετικό πρόσημο και το αποτέλεσμα (\texttt{\textlatin{sub\_res}}) έχει διαφορετικό πρόσημο από τον μειωτέο (\texttt{\textlatin{op1}}).
    \item \textbf{Πολλαπλασιασμός (\textlatin{MUL}):} Ο πολλαπλασιασμός 32*32 παράγει ένα ενδιάμεσο αποτέλεσμα 64-\textlatin{bit} (\texttt{\textlatin{mul\_res}}).
    Υπερχείλιση συμβαίνει εάν το αποτέλεσμα δεν χωράει σε 32 \textlatin{bits}, δηλαδή αν τα 32 ανώτερα \textlatin{bits} (\texttt{\textlatin{mul\_res[63:32]}}) δεν αποτελούν απλή επέκταση προσήμου (\textlatin{sign-extension}) του 31ου \textlatin{bit}.
\end{itemize}
Για όλες τις λογικές πράξεις και τις ολισθήσεις, η σημαία \texttt{\textlatin{ovf}} τίθεται πάντα σε \textlatin{1'b0}, καθώς η υπερχείλιση δεν ορίζεται για αυτές.
Για τις πράξεις ολίσθησης, αξιοποιήθηκε η συμπεριφορά της \textlatin{Verilog}:
\begin{itemize}
    \item \textbf{Αριθμητική Δεξιά Ολίσθηση} (\texttt{\textlatin{>>>}}): Επειδή ο \texttt{\textlatin{op1}} είναι \texttt{\textlatin{signed}}, ο τελεστής αυτός διατηρεί αυτόματα το πρόσημο (γεμίζει με το \textlatin{MSB}).
    \item \textbf{Λογική Δεξιά Ολίσθηση} (\texttt{\textlatin{>>}}): Απαιτήθηκε ρητή μετατροπή (\textlatin{casting}) του τελεστή σε \texttt{\textlatin{\$unsigned(op1)}} για να εξασφαλιστεί ότι η ολίσθηση θα γεμίσει τα κενά με μηδενικά, ανεξαρτήτως προσήμου.
    \item \textbf{Αριστερές Ολισθήσεις} (\texttt{\textlatin{<<}}): Η λογική και η αριθμητική αριστερή ολίσθηση είναι ταυτόσημες και γεμίζουν πάντα με μηδενικά.
\end{itemize}
Τέλος, η σημαία \texttt{\textlatin{zero}} υλοποιήθηκε εκτός του \texttt{\textlatin{always}} μπλοκ, με μια συνεχή ανάθεση (\texttt{\textlatin{assign zero = (result == 32'b0)}}).
Αυτό εξασφαλίζει ότι η έξοδος \texttt{\textlatin{zero}} είναι 1 οποτεδήποτε το τελικό \texttt{\textlatin{result}} είναι μηδέν.
Μια \texttt{\textlatin{default}} περίπτωση προστέθηκε στη δομή \texttt{\textlatin{case}} για την αποφυγή συμπερασμού \textlatin{latches}.
Ο πλήρης κώδικας της μονάδας παρατίθεται στο αρχείο \texttt{\textlatin{alu.v}}.

% 2.3
\subsection{\textlatin{Testbench} και Αποτελέσματα Προσομοίωσης}
Για την επαλήθευση της ορθής λειτουργίας της \textlatin{ALU}, δημιουργήθηκε ένα \textlatin{testbench} στο αρχείο \texttt{\textlatin{testbench\_alu.v}}.
Το \textlatin{testbench} αυτό εκτελεί μια σειρά από προκαθορισμένες δοκιμές μέσω ενός \texttt{\textlatin{initial block}}.
Χρησιμοποιήθηκε ένα βοηθητικό \texttt{\textlatin{task}} με όνομα \texttt{\textlatin{check\_op}}, το οποίο αναλαμβάνει να θέσει τις τιμές των εισόδων \texttt{\textlatin{tb\_op1}}, \texttt{\textlatin{tb\_op2}} και \texttt{\textlatin{tb\_alu\_op}}, να αναμένει 10\textlatin{ns} για τη διάδοση του αποτελέσματος στο συνδυαστικό κύκλωμα, και κατόπιν να εκτυπώσει στην κονσόλα την εκτελούμενη πράξη, τις εισόδους, το αποτέλεσμα (\texttt{\textlatin{tb\_result}}) και τις σημαίες (\texttt{\textlatin{tb\_ovf}}, \texttt{\textlatin{tb\_zero}}) σε δεκαεξαδική μορφή.
Οι δοκιμές κάλυψαν και τις 12 πράξεις, δίνοντας έμφαση σε οριακές περιπτώσεις:
\begin{itemize}
    \item Έλεγχος θετικής (\textlatin{MAX\_INT + 1}) και αρνητικής υπερχείλισης για την πρόσθεση (\textlatin{SUM}).
    \item Έλεγχος της σημαίας \texttt{\textlatin{zero}} με την αφαίρεση \texttt{25 - 25}.
    \item Έλεγχος υπερχείλισης για την αφαίρεση (\textlatin{MIN\_INT - 1}).
    \item Έλεγχος υπερχείλισης για τον πολλαπλασιασμό (\texttt{2\textsuperscript{16} * 2\textsuperscript{16}}).
    \item Σύγκριση μεταξύ λογικής και αριθμητικής δεξιάς ολίσθησης σε έναν αρνητικό αριθμό (\textlatin{32'hF000000A}), για να επιβεβαιωθεί η σωστή διατήρηση (ή μη) του προσήμου.
\end{itemize}
Η εκτέλεση του \textlatin{testbench} (όπως φαίνεται στην έξοδο της κονσόλας) επιβεβαίωσε ότι όλες οι πράξεις, συμπεριλαμβανομένων των οριακών συνθηκών, παράγουν τα αναμενόμενα αποτελέσματα και ότι οι σημαίες \texttt{\textlatin{ovf}} και \texttt{\textlatin{zero}} ενεργοποιούνται σωστά.

% 2.4
\subsection{Συμπέρασμα Άσκησης 1}
Η σχεδίαση της 32-\textlatin{bit} \textlatin{ALU} ολοκληρώθηκε με επιτυχία.
Η μονάδα που υλοποιήθηκε στο \texttt{\textlatin{alu.v}} είναι ένα αμιγώς συνδυαστικό κύκλωμα που εκτελεί σωστά και τις 12 απαιτούμενες αριθμητικές, λογικές και πράξεις ολίσθησης.
Οι μηχανισμοί ανίχνευσης υπερχείλισης και μηδενικού αποτελέσματος λειτουργούν όπως αναμένεται, σύμφωνα με την επαλήθευση που έγινε με το \textlatin{testbench}.
Αυτή η \textlatin{ALU} είναι πλέον έτοιμη να ενσωματωθεί στα πιο σύνθετα συστήματα των επόμενων ασκήσεων.


%% -- 3 -- %%
\newpage
\section{Άσκηση 2: Αριθμομηχανή \textlatin{16-bit}}

% 3.1
\subsection{Σκοπός και Προδιαγραφές}
Σκοπός της δεύτερης άσκησης ήταν η σχεδίαση μιας απλής αριθμομηχανής, η οποία χρησιμοποιεί την \textlatin{32-bit} \textlatin{ALU} που δημιουργήθηκε στην Άσκηση 1. Το κύκλωμα της αριθμομηχανής (\textlatin{module} \texttt{\textlatin{calc}}) σχεδιάστηκε για να διατηρεί μια τρέχουσα τιμή σε έναν καταχωρητή συσσωρευτή (\textlatin{accumulator}) \textlatin{16-bit}.

Οι είσοδοι του κυκλώματος είναι το σήμα ρολογιού (\texttt{\textlatin{clk}}), πέντε πλήκτρα ελέγχου (\texttt{\textlatin{btnc}}, \texttt{\textlatin{btnac}}, \texttt{\textlatin{btnl}}, \texttt{\textlatin{btnr}}, \texttt{\textlatin{btnd}}) και 16 διακόπτες (\texttt{\textlatin{sw}}) για την εισαγωγή δεδομένων. Η μοναδική έξοδος είναι 16 \textlatin{LED} (\texttt{\textlatin{led}}), τα οποία απεικονίζουν την τρέχουσα τιμή του συσσωρευτή.

% 3.2
\subsection{Σχεδίαση και Ροή Δεδομένων}
Η υλοποίηση (αρχείο \texttt{\textlatin{calc.v}}) βασίστηκε στο διάγραμμα ροής που δόθηκε στις προδιαγραφές. Το σύστημα αποτελείται από δύο βασικά μέρη: το ακολουθιακό κύκλωμα του συσσωρευτή και το συνδυαστικό κύκλωμα της \textlatin{ALU} και της λογικής ελέγχου.

\paragraph{Ακολουθιακή Λογική (\textlatin{Accumulator})}
Η καρδιά του συστήματος είναι ο \textlatin{16-bit} καταχωρητής \texttt{\textlatin{accumulator}}, ο οποίος υλοποιήθηκε με ένα μπλοκ \texttt{\textlatin{always @(posedge clk)}}. Η λειτουργία του είναι σύγχρονη:
\begin{itemize}
    \item \textbf{Σύγχρονος Μηδενισμός:} Όταν το πλήκτρο \texttt{\textlatin{btnac}} (\textlatin{All Clear}) είναι πατημένο, ο \texttt{\textlatin{accumulator}} μηδενίζεται στην επόμενη θετική ακμή του ρολογιού.
    \item \textbf{Σύγχρονη Φόρτωση:} Όταν το κεντρικό πλήκτρο \texttt{\textlatin{btnc}} είναι πατημένο (και το \texttt{\textlatin{btnac}} δεν είναι), ο \texttt{\textlatin{accumulator}} λαμβάνει και αποθηκεύει τα 16 κατώτερα \textlatin{bits} (\texttt{\textlatin{[15:0]}}) του αποτελέσματος της \textlatin{ALU} (\texttt{\textlatin{alu\_result\_wire}}).
\end{itemize}
Η τιμή του \texttt{\textlatin{accumulator}} οδηγείται συνεχώς στην έξοδο \texttt{\textlatin{led}} μέσω μιας \texttt{\textlatin{assign}} δήλωσης.

\paragraph{Συνδυαστική Λογική (Ροή \textlatin{ALU})}
Η \textlatin{ALU} της Άσκησης 1 είναι \textlatin{32-bit}, ενώ η αριθμομηχανή λειτουργεί με τιμές \textlatin{16-bit}. Για να γεφυρωθεί αυτό το χάσμα, χρησιμοποιήθηκε επέκταση προσήμου (\textlatin{sign extension}):
\begin{itemize}
    \item \textbf{Είσοδος \textlatin{op1}:} Η \textlatin{16-bit} τιμή του \texttt{\textlatin{accumulator}} μετατρέπεται σε \textlatin{32-bit} προσημασμένη τιμή (\texttt{\textlatin{op1\_signed}}) επαναλαμβάνοντας το ανώτερο \textlatin{bit} του (\texttt{\textlatin{accumulator[15]}}) 16 φορές. Αυτό υλοποιήθηκε με τον τελεστή \textlatin{concatenation}: \texttt{\{\{16\{accumulator[15]\}\}, accumulator\}}.
    \item \textbf{Είσοδος \textlatin{op2}:} Αντίστοιχα, η \textlatin{16-bit} είσοδος από τους διακόπτες (\texttt{\textlatin{sw}}) επεκτείνεται σε \textlatin{32-bit} προσημασμένη τιμή (\texttt{\textlatin{op2\_signed}}) για να οδηγηθεί στην \texttt{\textlatin{op2}} είσοδο της \textlatin{ALU}.
\end{itemize}
Το \textlatin{32-bit} αποτέλεσμα (\texttt{\textlatin{result}}) της \textlatin{ALU} τροφοδοτείται πίσω στην είσοδο του \texttt{\textlatin{accumulator}}, ο οποίος (όπως αναφέρθηκε) κρατά μόνο τα 16 κατώτερα \textlatin{bits}.

% 3.3
\subsection{Λογική Ελέγχου (\texttt{\textlatin{calc\_enc.v}})}
Η επιλογή της πράξης που εκτελεί η \textlatin{ALU} δεν γίνεται απευθείας. Αντ' αυτού, ένα ξεχωριστό \textlatin{module} κωδικοποιητή (\texttt{\textlatin{calc\_enc}}) μετατρέπει τις τιμές των τριών πλήκτρων κατεύθυνσης (\texttt{\textlatin{btnl}}, \texttt{\textlatin{btnr}}, \texttt{\textlatin{btnd}}) στο \textlatin{4-bit} σήμα ελέγχου \texttt{\textlatin{alu\_op}}.

Αυτό το \textlatin{module} υλοποιεί την συνδυαστική λογική που περιγράφηκε στα Σχήματα 2 έως 5 των προδιαγραφών. Αν και οι οδηγίες ζητούσαν η υλοποίηση να γίνει σε \textlatin{structural Verilog} (με χρήση πυλών \textlatin{AND}, \textlatin{OR}, \textlatin{NOT}), στην παρούσα υλοποίηση (ενσωματωμένη στο αρχείο \texttt{\textlatin{calc.v}}) επιλέχθηκε η περιγραφή της λογικής με συνεχείς αναθέσεις (\texttt{\textlatin{assign statements}}) για λόγους απλότητας και αναγνωσιμότητας.

% 3.4
\subsection{\textlatin{Testbench} και Αποτελέσματα Προσομοίωσης}
Για την επαλήθευση της ορθής λειτουργίας της αριθμομηχανής, δημιουργήθηκε το αρχείο \texttt{\textlatin{testbench\_calc.v}}. Το \textlatin{testbench} αυτό υλοποιεί την ακριβή ακολουθία δοκιμών που ορίζεται στον πίνακα των προδιαγραφών.

Το \textlatin{testbench} παράγει ένα σήμα ρολογιού (\texttt{\textlatin{CLK\_PERIOD = 10ns}}) και χρησιμοποιεί ένα βοηθητικό \texttt{\textlatin{task}} με όνομα \texttt{\textlatin{check\_step}}. Αυτό το \texttt{\textlatin{task}} είναι κρίσιμο για τον έλεγχο ακολουθιακών κυκλωμάτων:
\begin{enumerate}
    \item Θέτει τις τιμές όλων των εισόδων (πλήκτρα και διακόπτες).
    \item Αναμένει την επόμενη θετική ακμή του ρολογιού (\texttt{\textlatin{@ (posedge clk)}}), επιτρέποντας στον \texttt{\textlatin{accumulator}} να ενημερωθεί.
    \item Ελέγχει την έξοδο \texttt{\textlatin{led}} (μετά από μια μικρή καθυστέρηση \#1) για να διασφαλίσει ότι ταιριάζει με την αναμενόμενη τιμή (\texttt{\textlatin{expected\_led}}).
\end{enumerate}
Η ακολουθία ελέγχου ξεκινά με \texttt{\textlatin{RESET}} (\texttt{\textlatin{btnac=1}}) και εκτελεί τις πράξεις \textlatin{ADD}, \textlatin{XOR}, \textlatin{LSR}, \textlatin{NOR}, \textlatin{MULT}, \textlatin{LSL}, \textlatin{NAND} και \textlatin{SUB}, θέτοντας το \texttt{\textlatin{btnc=1}} σε κάθε βήμα για να φορτώσει το αποτέλεσμα. Η προσομοίωση επιβεβαίwσε ότι η έξοδος \texttt{\textlatin{led}} ήταν ίδια με την αναμενόμενη τιμή σε κάθε βήμα, επαληθεύοντας τη σωστή λειτουργία του κυκλώματος.

\subsection{Συμπέρασμα Άσκησης 2}
Η Άσκηση 2 ολοκληρώθηκε με επιτυχία, ενσωματώνοντας την συνδυαστική \textlatin{ALU} της Άσκησης 1 σε ένα πλήρες ακολουθιακό κύκλωμα. Η σχεδίαση διαχειρίζεται σωστά τη ροή δεδομένων μεταξύ στοιχείων \textlatin{16-bit} (όπως ο \texttt{\textlatin{accumulator}} και το \texttt{\textlatin{sw}}) και της \textlatin{32-bit} \textlatin{ALU} μέσω της επέκτασης προσήμου (\textlatin{sign extension}). Η λογική ελέγχου για τον σύγχρονο μηδενισμό και φόρτωση του συσσωρευτή, καθώς και η αποκωδικοποίηση των πλήκτρων, λειτούργησαν όπως αναμενόταν κατά την προσομοίωση.










\end{document}
